\subsection{Configuration de Ubidots}
Avant de se lancer dans la programmation, la Configuration de la plateforme Ubidots est nécessaire afin de préparer la hôte pour les données envoyées par le capteur.

Après une inscription, la première étape consiste à créer une nouveau composant comme le montre la figure \ref{newdevice}.

\begin{figure}[H]
  \centering
  \includegraphics[width=\textwidth]{imgs/newdevice.png}
  \caption{Description du capteur AD8232 \label{newdevice}}
\end{figure}

La plateforme offre une large variante de composants comme la montre la figure \ref{devices}

\begin{figure}[H]
  \centering
  \includegraphics[scale=.4]{imgs/devices.png}
  \caption{Description du capteur AD8232 \label{devices}}
\end{figure}

Maintenant, il faut attribuer une variable à ce composant comme le montre la figure \ref{addnewdevice}

\begin{figure}[H]
  \centering
  \includegraphics[scale=.4]{imgs/addnewdevice.png}
  \caption{Description du capteur AD8232 \label{addnewdevice}}
\end{figure}

Finalement, la variable \mintinline{c++}{ecg_val} est créé comme le montre la figure \ref{ecg_val}. Cette variable est utilisée par le programme afin de communiquer les données au capteur à la plateforme (voir l'appendix \ref{codesource}).

\begin{figure}[H]
  \centering
  \includegraphics[scale=.4]{imgs/ecg_val.png}
  \caption{Description du capteur AD8232 \label{ecg_val}}
\end{figure}

D'un autre coté, il faut ajouter un nouveau dashboard pour visualiser les données, comme le montre la figure \ref{dashboard}

\begin{figure}[H]
  \centering
  \includegraphics[scale=.4]{imgs/dashboard.png}
  \caption{Description du capteur AD8232 \label{dashboard}}
\end{figure}

Une grande variété de widgets est offerte afin de s'adapter aux différents besoins de l'utilisateur. Comme nous retraçons l'activité électirique cardiaque, nous voulons voir son évolution sous forme d'une courbe comme le montre la figure \ref{newwidget}

\begin{figure}[H]
  \centering
  \includegraphics[scale=.4]{imgs/newwidget.png}
  \caption{Description du capteur AD8232 \label{newwidget}}
\end{figure}