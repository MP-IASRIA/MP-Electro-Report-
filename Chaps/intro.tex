\chapter*{Introduction générale}
Les variétés de l'activité humaine aujourd'hui exige le suivi de l'état de santé à proximité. Ce qui s'est traduit par l'augmentation des équipements portés (wearable healthcare) de suivi du santé chaque année par 24\% chaque année\footnote{https://www.insiderintelligence.com/insights/wearable-technology-healthcare-medical-devices/} \cite{3}. La montre intélligente offre plusieurs services outre que le temps. La diversité de ces équipements prend plusieurs formes et assure la collecte de plusieurs métriques, la pression artérielle, concentration de $O_2$, \ldots

L'électrocardiogramme (ECG) est un test simple pour se renseigner rapidement sur le rythme et l'activité électrique du c\oe ur. C'est pourquoi l'analyse d'une telle information critique aide à conclure sur la normalité de l'activité cardiaque d'une façon générale. L'analyse du signal ECG, selon le type d'analyse, peut se faire sur plusieurs étapes, le traitement du signal, l'extraction des caractéristiques, les transformations et finalement la classification \cite{1, 2}.

Aujourd'hui, la connectivité peut améliorer l'acquisition et l'enregistrement des données. C'est l'ère des objets connectés où ces objets peuvent envoyer les informatios sur des plateformes dédiées. Ces plateformes offrent d'autres traitements sur les données collectées.

Nous voulons, dans ce projet, concevoir un système de suivi de l'activité cardiaque. La solution doit être connectée. Le présent rapport détaille les étapes de conception et de réalisation de la solution.