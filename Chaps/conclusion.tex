\addcontentsline{toc}{chapter}{Conclusion générale}
\chapter*{Conclusion générale}

Nous avons essayé, dans le cadre de ce projet, de s'adresser à problème sérieux à savoir le suivi de l'activité cardiologique humaine. Une contrainte imposée de plus en plus dans un environnement accidenté nécessitant la surveillance vigilante de l'état de santé.

Ainsi, nous proposons une approche basée sur IoT. Ce choix se justifie par le besoin de pouvoir consulter et exploiter les données collectées en temps réel. Nous avons utilisé le capteur AD8232 pour la détection de l'activité cardiaque. D'un autre coté la carte ESP32 offre l'avantage de se connecter directement au réseau (sans avoir ajouter un autre composant) avec prix raisonnable. Nous avons choisi la plateforme ubidots pour l'enregistrement des données.

Finalement, nous avons réussi à enregistrer les données dans le cloud. D'autres traitements peuvent s'appliquer par la suite telle que la détection d'anomalies cardiologiques. En effet, ajouter une couche intelligente permettra la détection au préalable des problèmes cardiologiques.
